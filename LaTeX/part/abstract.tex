% --------------------------------------------------------------------------------------
% Abstract (English)
% --------------------------------------------------------------------------------------
%\cleardoublepage
%\pdfbookmark[0]{Abstract}{abstract}
\chapter*{Abstract}
%%\ihead{Abstract}
\setlength{\parindent}{2pc}
\noindent In the past few years, due to the global energy crisis and rapid expansion of production requirement improving productivity and reducing energy consumption have received extensively focus. Therefore, manufacturing system which aims to improve operation control efficiency and results in energy sparing, has been widely studied. Steady state analysis of prodution systems has obtained broadly investigated. On the contrary, the trasient performance remained largely unexplored. This research mainly focus on system modeling, transient performance evaluation, and production line parameter optimization. Indeed, transient behavior of production systems has generally practical and theoretical implications. 

The main contribution of this work is to implement the mathematical models in a high-level programming language, python, and make a result analysis. Furthermore, we also designed a experiment to find a optimized buffer size to improve the transient performance of the production lines.


\vskip 0.75cm
\noindent\textbf{Keywords:} Geomatric Machine, Production Lines, mathmaticall model, performance evaluation, transient analysis.