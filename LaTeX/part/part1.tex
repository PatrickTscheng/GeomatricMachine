\chapter{Introduction}
\pagenumbering{arabic}
\setlength{\parindent}{2pc}
\label{intro}
% #######################################################################
\noindent 
Production system has been studied widely during the last 65 years \cite{papadopolous1993queueing}. A production system is an industrial system that describe a procedure to transform from different resources into useful products. In this process, producing units (human operators, industrial robots, cells, etc.) and resource handling devices (shelves, carts, holders, vehicles, etc.) connected with each others so that desired products can be produced. It is a very important part of manufacturing research and application. 

% Extensive research has been invested in develpoing for design, modeling, improvement, analysis and control of production systems (for instance, monographs \cite{buzacott1993stochastic, askin1993modeling, bonomi1987approximate, rao2000performance}). Despite the fact that in practice production systems may take several kinds of physical topologies, serial lines [see Figure \ref{Serial line.}] and assembly systems [see Figure \ref{Assymbly system.}] are the two most basic structures used in different manufacturing environments. In the literature, meanwhile, while serial production lines have been extensively investigated, assembly systems are been paid much less attention. Early research on assembly systems only considered the cases of multi-sequence-single-server, where different types of parts arrive at a single server in oder to be assembled together \cite{harrison1973assembly, bonomi1987approximate}. Inspired by these works, few three-server system with limited sequence capacities have been studied. In these papers, dual servers represent component part production, and other servers represent assembly operations.


\begin{figure*}[!h]
	\centering
	\subfigure[Serial line.]{
		\includegraphics{figures/serial_line.tikz}
		\label{Serial line.}}
	\subfigure[Assymbly system.]{
		\includegraphics{figures/assembly_system.tikz}	
	 	\label{Assymbly system.}}
	\caption{two production system}
	\label{Serial productionline and assembly system.}
\end{figure*}


The steady state behavior of production systems has been deeply studied in the last few years. Although it is often difficult to declare from the partial view that a production system is in steady state, the steady-state analysis approach is effective and accurate enough for manufacturing systems with large production capacity. The large production capacity allows the system to decay instantaneously to a negligible time compared to the global production run-time, and, therefore, allows it to use steady-state methods. Unfortunately, in addition to large-capacity manufacturing, there are a large number of mid- and small-capacity manufacturing systems in practice, which usually operate in a different method. In some these cases, one production line is usually able to produce several end products but can only produce one type of end product, equipment or special product at a time because of process. This usually lead to small- to medium-size production run-based operation based on the customer's order, where a production run contains only a specific amount of particular type of product. Obviously, when the size of the production run is relatively small, the steady-state approaches cannot provide an ideal and accurate analysis of system. In some industries, a production run somtimes means a batch.

Nevertheless, the transient period of the behavior in production system has received much less research attention because of its complexity and the still large numbers of unsolved problems in steady state production systems research. On the other hand, recent research \cite{li2009throughput} has proved that transient analysis has become one of the most important fields in production systems research. 

