\chapter{Introduction}
\pagenumbering{arabic}
\setlength{\parindent}{2pc}
\label{intro}
% #######################################################################
\noindent 
Production system has been studied widely during the last 65 years \cite{papadopolous1993queueing}. A production system is an industrial system that describe a procedure to transform from different resources into useful products. In this process, producing units (human operators, industrial robots, cells, etc.) and resource handling devices (shelves, carts, holders, vehicles, etc.) connected with each others so that desired products can be produced. It is a very important part of manufacturing research and application. 

Extensive research has been invested in develpoing for design, modeling, improvement, analysis and control of production systems (for instance, monographs \cite{buzacott1993stochastic, askin1993modeling, bonomi1987approximate, rao2000performance}). Despite the fact that in practice production systems may take several kinds of physical topologies, serial lines [see Figure \ref{Serial line.}] and assembly systems [see Figure \ref{Assymbly system.}] are the two most basic structures used in different manufacturing environments. In the literature, meanwhile, while assembly systems have been extensively investigated, serial production lines are been paid much less attention. Early research on assembly systems only considered the cases of multi-sequence-single-server, where different types of parts arrive at a single server in oder to be assembled together \cite{harrison1973assembly, bonomi1987approximate}. Inspired by these works, few three-server system with limited sequence capacities have been studied \cite{kuo1996improvability,lipper1986assembly}. In these papers, dual servers represent component parts production, while the other server describes assembly operations. In addition, assembly systems based on queueing model have been further explored in papers \cite{manitz2008queueing, rao2000performance, rao1994approximate} and the references within . The problem of steady-state performance evaluation of assembly system with unreliable machines and limited buffers has been studied in works \cite{gershwin1991assembly,liu1990approximate, helber1998decomposition, mascolo1991modeling, chiang2000improvability, chiang1970improvability} . In particular, the literature \cite{gershwin1991assembly} developed a deconstruction technique to approximate the steady-state throughput for assembly systems based on machines with geometric models and identical processing times, while the article \cite{liu1990approximate} focuses on assembly systems with geometric machines and a nonidentical processing time through turning the assembly system to a serial line. Furthermore, paper \cite{helber1998decomposition} develops the analysis to assembly system by geometric processing times.

\begin{figure*}[!ht]
	\centering
	\subfigure[Serial line.]{
		\includegraphics{figures/serial_line.tikz}
		\label{Serial line.}}
	\subfigure[Assymbly system.]{
		\includegraphics{figures/assembly_system.tikz}	
	 	\label{Assymbly system.}}
	\caption{two production system}
	\label{Serial productionline and assembly system.}
\end{figure*}


The steady state behavior of production systems has been deeply studied in the last few years \cite{chiang2000improvability, chiang1970improvability}. Although it is often difficult to declare from the partial view that a production system is in steady state, the steady-state analysis approach is effective and accurate enough for manufacturing systems with large production capacity. The large production capacity allows the system to decay instantaneously to a negligible time compared to the global production run-time, and, therefore, allows it to use steady-state methods. Unfortunately, in addition to large-capacity manufacturing, there are a large number of mid- and small-capacity manufacturing systems in practice, which usually operate in a different method. In some these cases, one production line is usually able to produce several end products but can only produce one type of end product, equipment or special product at a time because of process. This usually lead to small- to medium-size production run-based operation based on the customer's order, where a production run contains only a specific amount of particular type of product. Obviously, when the size of the production run is relatively small, the steady-state approaches cannot provide an ideal and accurate analysis of system. In some industries, a production run somtimes means a batch.

Nevertheless, the transient period of the behavior in production system has received much less research attention because of its complexity and the still large numbers of unsolved problems in steady state production systems research. On the other hand, recent research \cite{li2009throughput} has proved that transient analysis has become one of the most important fields in production systems research. Indeed, transient behavior of production systems have not been systematically studied and it is considered as one of the most significant directions in production systems research \cite{li2009throughput}. Specially, transient properties of serial production systems with two machines having the Bernoulli reliability model have been explored in \cite{meerkov2008transient, meerkov2009transients, meerkov2011unbalanced} based in Markovian analysis. The research was later also extended to the case of multi-machine Bernoulli lines in \cite{zhang2013transient, chen2012energy}, which set up computationally efficient algorithms with recursive aggregation to approximate the transient performance with high accuracy. 

Applications of Bernoulli line transient properties analysis is reported in \cite{chen2012energy, wang2010transient, chen2011feedback}. Specailly, the paper \cite{chen2012energy} extended the algorithm developed in work\cite{zhang2013transient} to the Bernoulli series production line with time-varying machine parameters. In spite of important results have been obtained regarding the transient behabior of the production system, it should be considered that most of the analysis studies cited above are only applicable to systems with machines based on Bernoulli reliability model, which can only be applied in the situations where the average machine downtime is comparable to its cycle time. Although the paper \cite{meerkov2010transient} tried to study the transient properties of serial production lines with machines in the geometric reliability model, the results were only applicable to the case of two-machine lines with an initial buffer occupancy at the beginning. For two-machine production lines with general initial state and longer lines, as far as we know, no analytical methods for have been built up for analysis of their transient performance because of larger dimension of the system state. Thus, the goal of this paper is to set up mathmaticall models in python code with the help of the Bernoulli reliabile geomatric machine serial production line. Then, compare the results and consider the causes of the differences. In addition, an analysis of the relationship between the parameter improvement and  trasient performance is evaluated.

The remainder of the papaer is organized as follows: Section II introduces the assumptions for the system and presents the performance measures of interest, and theory of Markov chain. mathmaticall modeling and behavior performance evaluation of individual machines, two-machine lines. and multi-machine lines are described in Section III. Nevertheless, the optimazition of a serial production line is given in Section IV. The conclusions and future work are given in Section V.